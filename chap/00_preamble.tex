%****************************************
% 00_preamble
%****************************************

% import Latex packages
\usepackage{ifthen}

\usepackage[T1]{fontenc}
\usepackage[utf8]{luainputenc}

% \includegraphics
\usepackage{graphicx}
% Bilder/Tabellen drehen
\usepackage{rotating}
% Rotatebox
\usepackage[export]{adjustbox}
\usepackage{graphics}
% Mathematikpaket
\usepackage{amsmath}
% To write vectors: \vb
\usepackage{physics}
% für die Auswahl von Schriften
\usepackage{fontspec}

% language packages
\usepackage[english, ngerman]{babel}
\usepackage{blindtext}
% für spezifisches Layout 
\usepackage{geometry}
\usepackage{layout}
\usepackage{setspace}
% "ansehliche" Tabellen
\usepackage{booktabs}
% erweiterte Möglichkeiten für die Positionierung von Bildern, Tabellen, ..
\usepackage{float}
% \FloatBarrier setzen
\usepackage{placeins}
% Blocksatz verbessern bei pdf Erzeugung
\usepackage{microtype} 	
% Kapitelüberschrift bei Querverweisen anzeigen
\usepackage{nameref}
% Ermöglichen, Grafiken nebeneinander und innerhalb einer Grafik zu platzieren
\usepackage{subfig}				
% Einfügen von PDF Dateien
\usepackage{pdfpages}
% Einfügen von SVG images
\usepackage{svg}
% Anmerkungen im Text
%\usepackage{todonotes}
% für Tabellen
\usepackage{array}
% Farbdefinitionen
\usepackage{xcolor}
\definecolor{LightGray}{gray}{0.9}
% für Code
\usepackage[outputdir=auxil]{minted}
\setminted[python]{
    frame=lines,
    framesep=2mm,
    bgcolor=LightGray,
    fontsize=\footnotesize,
    linenos
}
%\renewcommand{\listingscaption}{Program code}
%\renewcommand{\listoflistingscaption}{List of Program Code}
%\newcommand{\pylisting}[2]{
%\begin{listing}[H]
%\begin{minted}{python}
%#2
%\end{minted}
%\caption{#1}
%%\label{lst:example}
%\end{listing}
%}

% PDF
%\usepackage[unicode=true,pdfusetitle,bookmarks=true,bookmarksnumbered=false,bookmarksopen=false,breaklinks=true,pdfborder={0 0 1},backref=false]{hyperref}
% Abkürzungen
\usepackage[printonlyused]{acronym}
% Zeilenumbrüche in url
%\usepackage{luacode}
%\luaexec{require("lua/betterurl")}
%\renewcommand{\url}[1]{\luadirect{
%    betterurl(\luastringN{#1})
%}}
% Alternative, da obiges keine underscores unterstützt
\usepackage{hyperref}
\urlstyle{same}

% Kopf- und Fußzeile
\usepackage{fancyhdr}

% kleinere Überschriften
\KOMAoptions{headings=small}
% Konfiguration des Inhaltsverzeichnisses
\KOMAoptions{toc=listof}
\KOMAoptions{toc=bib}

% Carlito (Unix) or Calibri as default font
\setmainfont{\defaultFont{}}

% Kein Einrücken der ersten Zeile eines Absatzes, dafür
% Kleiner Abstand zwischen den Absätzen
%\setlength{\parskip}{\smallskipamount}
\setlength{\parindent}{0pt}

% keine Einrückung der Texte bei Bildunterschriften
\usepackage{caption} 
\captionsetup{format=plain} 

\makeatletter

% Alle Nummerierungen kapitelweise (chapter)
\numberwithin{figure}{chapter}
\numberwithin{equation}{chapter}
\numberwithin{table}{chapter}

% Inhaltsverzeichnis Tiefe
%\setcounter{secnumdepth}{2}
%\setcounter{tocdepth}{2}

% Aktuelles Datum initialisieren (falls benötigt)
\@ifundefined{date}{}{\date{}}

\makeatother

% Definition der Ränder
%\geometry{paperheight=297mm, 
%	paperwidth=210mm,
%	top=25mm,
%	textwidth=155mm,
%	textheight=252mm,
%	right=25mm,
%	heightrounded,
%	headheight=5mm,
%	headsep=10mm,
%	foot=15mm,
%	marginparsep=0mm,
%	marginparwidth=20mm
%}
\geometry{left=3.5cm,
    right=3cm,
    top=3cm,
    bottom=3cm
}


%Quotation marks:
%„ -> ALT 0132
%“ -> ALT 0147